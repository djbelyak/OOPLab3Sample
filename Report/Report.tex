\documentclass[a4paper,12pt]{article}
\usepackage[T2A]{fontenc}  %поддержка кириллицы в ЛаТеХ
\addtolength{\hoffset}{-1.7mm} % горизонтальное смещение всего текста как целого
\usepackage[utf8]{inputenc}  %По умолчанию кодировка KOI8 для *nix-систем
\usepackage[english,russian]{babel} %определение языков в документе
\usepackage{amssymb,amsmath,amsfonts,latexsym,mathtext} %расширенные наборы
  % математических символов
\usepackage{cite}  %"умные" библиографические ссылки
%(сортировка и сжатие)
\usepackage{indentfirst} %делать отступ в начале параграфа
\usepackage{enumerate}  %создание и автоматическая нумерация списков
\usepackage{tabularx}  %продвинутые таблицы
%\usepackage{showkeys}  %раскомментируйте, чтобы в документе были видны
%ссылки на литературу, рисунки и таблицы
\usepackage[labelsep=period]{caption} %заменить умолчальное разделение ':' на '.'
% в подписях к рисункам и таблицам
%\usepackage[onehalfspacing]{setspace} %"умное" расстояние между строк - установить
% 1.5 интервала от нормального, эквивалентно
 \renewcommand{\baselinestretch}{1.24}
\usepackage{graphicx} %разрешить включение PostScript-графики
\graphicspath{{Images/}} %относительный путь к каталогу с рисунками,это может быть мягкая ссылка
\usepackage{listings}


\RequirePackage{caption}
\DeclareCaptionLabelSeparator{defffis}{ -- }
\captionsetup{justification=centering,labelsep=defffis}

\usepackage{geometry} %способ ручной установки полей
\geometry{top=2cm} %поле сверху
\geometry{bottom=2cm} %поле снизу
\geometry{left=2cm} %поле справа
\geometry{right=1cm} %поле слева

\usepackage[colorlinks,linkcolor=blue]{hyperref}%гиперссылки в тексте
\newcommand{\tocsecindent}{\hspace{7mm}}% отступ для введения

\makeatletter
\bibliographystyle{unsrt} %Стиль библиографических ссылок БибТеХа - нумеровать
%в порядке упоминания в тексте
\renewcommand{\@biblabel}[1]{#1.}
\makeatother

\begin{document}

%Титулный лист
%Титульный лист пояснительной записки

\begin{titlepage}
\newpage

\begin{center}
{\small\bfМИНИСТЕРСТВО ОБРАЗОВАНИЯ И НАУКИ РОССИЙСКОЙ ФЕДЕРАЦИИ\\
ОБНИНСКИЙ ИНСТИТУТ АТОМНОЙ ЭНЕРГЕТИКИ --- филиал}\\
федерального государственного автономного образовательного учреждения\\
высшего профессионального образования\\
{\bf<<Национальный исследовательский ядерный университет <<МИФИ>>\\
(ИАТЭ НИЯУ МИФИ)}\\
\vspace{2em}
Факультет кибернетики\\
Кафедра компьютерных систем, сетей и технологий
\end{center}
\vspace{2em}


\vspace{5em}
\begin{center}
\textbf{ОТЧЕТ ПО ЛАБОРАТОРНОЙ РАБОТЕ №3\\ Тут предметная область}
\end{center}

%\vspace{2em}

\begin{center}
По курсу <<Объектно-ориентированное программирование>>
\end{center}

\vspace{6em}

\hbox to \textwidth
{\parbox{6 cm}{Студент гр. ВТ-Б12}\dotfill \parbox{4 cm}{
\begin{flushright}Иванов~И.И.\end{flushright}}}
\vspace{2em}

\hbox to \textwidth
{\parbox{6 cm}{Руководитель\\доцент\\ кафедры КССТ}\dotfill \parbox{4 cm}{
\begin{flushright}Тельнов~В.П.\end{flushright}}}
\vspace{2em}



\vspace{\fill}

\begin{center}
Обнинск 2014
\end{center}

\end{titlepage}


\setcounter{page}{2} % начать нумерацию с номера три
\renewcommand{\figurename}{Рисунок}


\newpage
\section{Постановка задачи}

Спроектируйте в Visual Paradigm и реализуйте объектно-ориентированное приложение на языке С++ (C\#, Java), которое представляет собой упрощенную, но действующую программную модель некоторой системы или процесса реального мира.
Приложение снабдите надлежащим пользовательским интерфейсом с тем, чтобы можно было задавать исходные данные для моделирования, наблюдать за ходом процесса и просматривать итоговые результаты. 

Список предметных областей для моделирования: Лифт; Автосервис; Театральная касса; Аптека; Книжный магазин; Деканат; Касса универсама; Регистратура поликлиники; Магазин компьютерной техники; Бензозаправочная станция;  Центр мобильной связи;  Компьютерный класс; Библиотека; Маршрутное такси; Интернет-магазин; Туристическое агентство. 
Список является открытым, можно предлагать собственные объекты, системы, процессы. 

Для решения задачи: 
\begin{enumerate}
\item  Изучите нотации UML (Unified Modeling Language), которые применяются для построения диаграмм прецедентов и диаграмм классов.

\item Осуществите анализ и объектную декомпозицию выбранной предметной области  (системы, процесса), применив разумные упрощения. 

\item Спроектируйте в Visual Paradigm диаграммы прецедентов и диаграммы классов, обсудите их в проектной группе, а затем с преподавателем. 

\item Сгенерируйте в Visual Paradigm необходимые классы и интерфейсы, перенесите их в проект. 

\item Доработайте приложение и отладьте полученный код.
Для организации коллективной работы используйте GitHub. 
Убедитесь, что созданное приложение моделирует предметную область правдоподобным образом. 
\end{enumerate}

\subsection{Рекомендации}

Начните работу с пробных проектов, чтобы поэкспериментировать с Visual Paradigm и языком программирования.
Изучите предложенный преподавателем проект на GitHub.
Обратите внимание, что многие предметные области могут быть описаны моделями, которые изучаются в курсе <<Моделирование систем>>. 
Активно применяйте контейнеры для работы с данными. 
Для моделирования случайных событий используйте генераторы случайных чисел. 
Возникающие затруднения пытайтесь преодолеть самостоятельно, потом обращайтесь за помощью. 

Письменный отчет по работе должен содержать следующие разделы: 
\begin{enumerate}
\item Постановку задачи. 

\item Описание предметной области, использованных упрощений и ограничений. 

\item Диаграммы прецедентов и диаграммы классов (отчет Visual Paradigm). 

\item Исходные данные и результаты моделирования предметной области. 

\item Листинг разработанного авторского кода. Код должен быть надлежащим образом структурирован и снабжен комментариями. 
\end{enumerate}

Для успешной сдачи лабораторной работы необходимо представить письменный отчет (один от всей проектной группы), продемонстрировать на практике работоспособность программного решения и каждому из студентов ответить на вопросы преподавателя.


\section{Описание предметной области}

%Опишите здесь предметную область

\section{UML-диаграммы}

%Для вставки картинок рнаскомментируйте окружение figure и заполните нужные поля

%\begin{figure}[h]
%\center{\includegraphics]{имя файла в папке Images}}
%\caption{Подпись рисунка.}
%\label{метка рисунка для ссылок}
%\end{figure}

\section{Исходные данные и результаты моделирования}

%Здесь требуется указать начальные условия моделирования и его итоги

\section{Листинг разработанного авторского кода}

%Здесь надо указать ссылки на все файлы с кодов в проекте

%lstinputlisting[language=Язык программирования]{Относительный путь к файлу}

\end{document}
\section{Постановка задачи}

Спроектируйте в Visual Paradigm и реализуйте объектно-ориентированное приложение на языке С++ (C\#, Java), которое представляет собой упрощенную, но действующую программную модель некоторой системы или процесса реального мира.
Приложение снабдите надлежащим пользовательским интерфейсом с тем, чтобы можно было задавать исходные данные для моделирования, наблюдать за ходом процесса и просматривать итоговые результаты. 

Список предметных областей для моделирования: Лифт; Автосервис; Театральная касса; Аптека; Книжный магазин; Деканат; Касса универсама; Регистратура поликлиники; Магазин компьютерной техники; Бензозаправочная станция;  Центр мобильной связи;  Компьютерный класс; Библиотека; Маршрутное такси; Интернет-магазин; Туристическое агентство. 
Список является открытым, можно предлагать собственные объекты, системы, процессы. 

Для решения задачи: 
\begin{enumerate}
\item  Изучите нотации UML (Unified Modeling Language), которые применяются для построения диаграмм прецедентов и диаграмм классов.

\item Осуществите анализ и объектную декомпозицию выбранной предметной области  (системы, процесса), применив разумные упрощения. 

\item Спроектируйте в Visual Paradigm диаграммы прецедентов и диаграммы классов, обсудите их в проектной группе, а затем с преподавателем. 

\item Сгенерируйте в Visual Paradigm необходимые классы и интерфейсы, перенесите их в проект. 

\item Доработайте приложение и отладьте полученный код.
Для организации коллективной работы используйте GitHub. 
Убедитесь, что созданное приложение моделирует предметную область правдоподобным образом. 
\end{enumerate}

\subsection{Рекомендации}

Начните работу с пробных проектов, чтобы поэкспериментировать с Visual Paradigm и языком программирования.
Изучите предложенный преподавателем проект на GitHub.
Обратите внимание, что многие предметные области могут быть описаны моделями, которые изучаются в курсе <<Моделирование систем>>. 
Активно применяйте контейнеры для работы с данными. 
Для моделирования случайных событий используйте генераторы случайных чисел. 
Возникающие затруднения пытайтесь преодолеть самостоятельно, потом обращайтесь за помощью. 

Письменный отчет по работе должен содержать следующие разделы: 
\begin{enumerate}
\item Постановку задачи. 

\item Описание предметной области, использованных упрощений и ограничений. 

\item Диаграммы прецедентов и диаграммы классов (отчет Visual Paradigm). 

\item Исходные данные и результаты моделирования предметной области. 

\item Листинг разработанного авторского кода. Код должен быть надлежащим образом структурирован и снабжен комментариями. 
\end{enumerate}

Для успешной сдачи лабораторной работы необходимо представить письменный отчет (один от всей проектной группы), продемонстрировать на практике работоспособность программного решения и каждому из студентов ответить на вопросы преподавателя.
